\documentclass[12pt]{article}
%Gummi|05|=)
\title{\textbf{EV\_2\_3\_Explicar\_los\_arreglos\_y\_parametros\_de\_
los\_parametros\_de\_los\_Amplificadores\_Clase\_B}}
\author{Osvaldo Romero Jauregui\\
		4-A Mecatronica\\
		Sistemas Electronicos de Interfaz}
\date{08/Octubre/2019}
\begin{document}

\maketitle

\section{Amplificadores Clase B}

Los amplificadores de clase B se caracterizan por tener intensidad casi nula a traves de sus transistores cuando no hay señal en la entrada del circuito, por lo que en reposo el consumo es casi nulo.

Se les denomina amplificador clase B, cuando el voltaje de polarizacion y la maxima amplitud de la señal entrante possen valores que hacen que la corriente de salida circule durante el semiciclo de la señal de entrada. La caracteristica principal de este tipo de amplificadores es el alto factor de amplificacion.

Los amplificadores de Clase B usan ds o mas transistores poarizados de tal forma que cada transistor solo conduce duranto uun medio ciclo de la onda de entrada.
Para mejorar la eficiencia de potencia total del amplificador Clase A previo reduciendo la potencia desperdiciada en forma de calor, es posible que diseñar eñ corcito amplificador de potencia con dos transistores en su etapa de salida produciendo lo que comunmente se denomina amplificador de clase B, tambien conocido como una configuracion de amplificador push-pull.

Los amplificadores push-pull utilizan dos transistores "complementarios" o coincidentes, uno de tipo NPN y el otro de tipo PNP con ambos transistores de potencia que reciben la misma señal de entrada que es igual en magnitud, pero en fase opuesta entre si. Esto resulta en un transistor solamente amplificar la mitad o 180 o del ciclo de forma de entrada con las resultantes de "dos mitades" esta poniendo juntos de nuevo en la salida terminal.

\end{document}
