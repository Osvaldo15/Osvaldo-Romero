\documentclass[11pt]{article}
%Gummi|065|=)
\title{\textbf{EV\_2\_4\_giro\_de\_un\_motor\_de\_corriente\_directa}}
\author{Osvaldo Romero Jauregui\\
		4-A Mecatronica\\
		Sistemas Electronicos de Interfaz}
\date{15/Octubre/2019}
\begin{document}

\maketitle

\section{Motor de Corriente Continua}
Un motor elecrico de Corriente Continua es esencialmene una maquina que convierte energia electrica en movimiento o trabajo mecanico, a traves de medios electromagneticos.
\textbf{Fundamentos de operacion de los motores electricos}
En magnetismo se conce la existemcia de dos polos: polo norte (N) y polo sur (S), que son las regiones donde concentran las lineas de fuerza de un iman. Un motor para funcionar se vale de las fuerzas de atraccion y repulsion que existen entre los polos. De acuerdo con esto, todo motor tiene que estar formado con polos alternados entre el estator y el rotor, ya que los polos magneticos iguales se repelen, y los polos magneticos diferentes se atraen, produciendo asi el movimiento de rotacion.

\section{Giro de motor}
Un moor cambia de sentido de giro cuando cambia la polaridad en sus bornes (contactos).
U  oor funciona en base a dos principios: El de induccion, que señala, que si un conductor se mueve a traves de un campo magnetico o esta situado en las proximidades de otro conductor por el que no circula una corriente de intensidad variable, se induce una corriente electrica en el primer conductor. Y el principio de Ampere que establece: Que si una corriente pasa a traves de un conductor situado en el interior de un campo magnetico, este ejerce una fuerza mecanica o f.e.m (fuerza electromotriz), sobre el conductor

\footnote{https://www.mecatronicalatam.com/motores/motores-de-cd}



\end{document}
