\documentclass[11pt]{article}
%Gummi|065|=)
\usepackage[spanish, activeacute]{babel} %Definir idioma español
\usepackage[utf8]{inputenc} %Codificacion utf-8
\usepackage{amssymb, amsmath, amsbsy} % simbolitos
\usepackage{upgreek} % para poner letras griegas sin cursiva
\usepackage{cancel} % para tachar
\usepackage{mathdots} % para el comando \iddots
\usepackage{mathrsfs} % para formato de letra
\usepackage{stackrel} % para el comando \stackbin
\usepackage{graphicx}

\title{\textbf{E\_V\_2\_8 Calcular los parametros de circuitos de activacion de transistores de potencia}}
\author{Osvaldo Romero Jauregui\\
		4-A Mecatrónica\\
		Sistemas Electrónicos de Interfaz}
\date{29/Octubre/2019}
\begin{document}

\maketitle

\section{Resistencia Base de un Transistor}
Un capacitor tiene la capacidad de ser analógico o digital, es decir, el analógico funciona como amplificador de onda y el dígital se utiliza como un switch, para información de ceros y unos en todo caso si queremos trabajar como amplificador se debe utilizar la polarización. Pero este al no se el caso queremos usar como digital, es decir; corte y saturación.

Como queremos usarla de manera digital debemos entender ¿cómo funciona un transistor?, para poder calcular su activación, un transistor npn tiene una tensión, base, colector, esta tensión dependiendo del material del transistor (germanio 0.3 y silicio 0.7) si no es superada es decir: VBE<0.7V entonces tendremos un switch cerrado, pero si encontramos que es mayor o mucho mayor es decir: VBE> 0.7V || VBE > 0.7 entonces se comportará como un switch cerrado y por tanto podremos ver que la corriente es = 0 en la base y el colector, sin embargo encontramos que la corriente en la base es mayor a cero. Asis podemos ver en palabras simples un paso o no paso.


\end{document}
